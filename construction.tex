\section{Construction}

To be written...

\newcommand{\RecursiveHash}{\ms{RecursiveHash}}
\newcommand{\RecursiveVerify}{\ms{RecursiveVerify}}
\newcommand{\croot}{c_{\ms{root}}}
\newcommand{\cleft}{c_{\ms{left}}}
\newcommand{\cright}{c_{\ms{right}}}
\newcommand{\midi}{\ms{mid}}
\newcommand{\RecursiveProve}{\ms{RecursiveProve}}
\newcommand{\tagleft}{\ms{L}}
\newcommand{\tagright}{\ms{R}}

\begin{construction}
  \label{cons:main}
  Let $\calD = \calD_{\lambda \in \N}$ be a domain space of inputs, which
  includes the set of positive integers $\N \subseteq \calD$. Let $\calH_\calK =
  \{ H_k: \calD \times \calD \to \calD\}_{k \in \calK}$ for a key space
  $\calK_{\lambda \in \N}$. We construct an accumulator scheme $\Piacc =
  (\Commit, \Prove, \Verify)$ with domain $\calX$, commitment space $\calY$, and
  proof space $\calD^*$ for $k \getsr \calK$ as follows:
  \begin{itemize}
    \item $\Commit(1^\lambda, x_0, \ldots, x_{\ell-1}) \to c$: On input the
      security parameter $\lambda$, sequence of inputs $x_0, \ldots,
      x_{\ell-1}$, the commit algorithm hashes the sequence of inputs using
      the recursive algorithm $\croot \gets \RecursiveHash(x_0, \ldots,
      x_{\ell-1})$ that is defined as follows:
      \begin{description}
        \item $\RecursiveHash(x_0, \ldots, x_{\ell-1})$:
          \begin{itemize}
            \item If $\ell = 1$, then simply return $x_0$.
            \item Otherwise, let $\midi \gets \lceil (\ell-1)/2 \rceil$. The
              function recursively computes $\cleft \gets \RecursiveHash(x_0,
              \ldots, x_\midi)$, $\cright \gets \RecursiveHash(x_{\midi + 1},
              \allowbreak x_{\ell-1})$, and then returns $H_k(\allowbreak \cleft,
              \allowbreak \cright)$.
          \end{itemize}
      \end{description}
      The commit algorithm finally computes $c \gets H_k(\croot, \ell)$ and return
      $c$ as the commitment.

    \item $\Prove(1^\lambda, x_0, \ldots, x_{\ell-1}, i) \to u$: On input the security
      parameter $\lambda$, set of inputs $x_0, \ldots, x_{\ell-1}$, and an index
      $i \in [\ell]$, the proving algorithm proceeds as follows. It maintains a
      set $S \subset \calD \times \{\tagleft, \tagright\}$ of pairs of domain
      elements in $\calD$ and tag $\{\tagleft, \tagright\}$ that is initially
      empty $S = \{ \}$. The algorithm adds elements to the set $S$ using the
      recursive algorithm $\RecursiveProve(S, x_0, \ldots, x_{\ell-1}, i)$ as
      follows:
      \begin{description}
        \item $\RecursiveProve(S, x_0, \ldots, x_{\ell-1}, i)$:
          \begin{itemize}
            \item If $\ell = 1$, then simply return.
            \item Otherwise, let $\midi \gets \lceil (\ell-1)/2 \rceil$
              \begin{itemize}
                \item If $i \le \midi$, then compute $\cright \gets
                  \RecursiveHash(x_{\midi + 1}, \ldots, x_{\ell-1})$ and add $S
                  \gets S \cup \{(\cright, \tagright)\}$. Then recursively compute
                  $\RecursiveProve(S, x_0, \ldots, x_\midi, i)$.
                \item If $i > \midi$, then compute $\cleft \gets
                  \RecursiveHash(x_0, \ldots, x_\midi)$ and add $S \gets S \cup
                  \{(\cleft, \tagleft)\}$. Then recursively compute
                  $\RecursiveProve(S, x_{\midi+1}, \ldots, x_{\ell-1}, i -
                  \midi)$.
              \end{itemize}
          \end{itemize}
          The proving algorithm sets $u \gets (S, \ell)$ and return $u$ as
          the proof.
      \end{description}

    \item $\Verify(c, x, u) \to b$: On input a commitment $c \in \calD$, input
      $x \in \calD$, and proof $u = (S, \ell)$, the verification algorithm
      first parses the proof $S = \{ (c_1, \tau_1), \ldots, (c_t, \tau_t) \}$
      for some $t \in \N$. Then it sets $h \gets x$ and invokes the recursive
      algorithm $\croot \gets \RecursiveVerify(h, (c_1, \tau_1), \ldots, (c_t,
      \tau_t))$ that is defined as follows:
      \begin{description}
        \item $\RecursiveVerify(h, (c_1, \tau_1), \ldots, (c_i, \tau_i))$:
          \begin{itemize}
            \item If $\tau_i = \tagleft$, then set $h \gets H_k(h, c_i)$ and
              recursively compute $\RecursiveVerify(h, \allowbreak (c_1,
              \tau_1), \allowbreak \ldots, (c_{i-1}, \tau_{i-1}))$.
            \item If $\tau_i = \tagright$, then set $h \gets H_k(c_i, h)$ and
              recursively compute $\RecursiveVerify(h, \allowbreak (c_1,
              \tau_1), \allowbreak \ldots, (c_{i-1}, \tau_{i-1}))$.
          \end{itemize}
      \end{description}
      Finally, the verification algorithm computes and verifies that $H_k(\croot,
      \ell) = c$. If the equality holds, the verification algorithm accepts
      (returns 1) and otherwise, it rejects (returns 0).

  \end{itemize}
\end{construction}

\begin{theorem}[Correctness]
    The accumulator scheme $\Piacc$ in Construction~\ref{cons:main} satisfies
    \emph{correctness} (Definition~\ref{def:correctness}).
\end{theorem}

\begin{theorem}[Compactness]
    The accumulator scheme $\Piacc$ in Construction~\ref{cons:main} satisfies
    \emph{compactness} (Definition~\ref{def:compactness}).
\end{theorem}

\begin{theorem}[Soundness]
    To be written...
\end{theorem}
